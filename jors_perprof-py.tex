\documentclass[10pt,a4paper]{article}
\usepackage[utf8]{inputenx}
\usepackage{cmap} 
\usepackage[T1]{fontenc}
\usepackage[english]{babel}
\usepackage[top=3cm,bottom=3cm,outer=3.2cm,inner=3.2cm]{geometry}
\usepackage[backend=biber,sortcites=true,doi=false,url=false,firstinits=true,hyperref,maxbibnames=9,maxcitenames=3,sorting=nyt]{biblatex}

\addbibresource{refs.bib}
\usepackage{csquotes}
\usepackage{url}
\usepackage{hyperref}
\usepackage{amsfonts}
\usepackage{amssymb}
\usepackage{amsthm}
\usepackage{mathtools}
\usepackage{dsfont}       %Stroke Fonts for Number Sets
\usepackage{cases}




\begin{document}
\title{Perprof-py: a {P}ython package for performance profile}
\author{Raniere Gaia Costa da Silva \and Abel Soares Siqueria  \and Luiz-Rafael Santos}
\date{Technical Report \\ Operational Research and Optimization Laboratory (LPOO) \\ IMECC/Unicamp \\ Campinas, Brazil \\ \today}
\maketitle

\begin{abstract}
Benchmarking optimization packages are very important in optimization field,
not only because it is one of the way to compare algorithms, but also to uncover
deficiencies that could be overlooked while one is developing new solvers. During
benchmarking, one can obtain several informations, like CPU time, number of functions
evaluations, number of iterations and so on. These informations, if presented
as tables, can be difficult to be analyzed, due, for instance, to large amount of data.
Therefore, researchers started testing tools to better process and understand this
data. One of the most widespread ways to do so is using Performance Profile graphics
proposed by \textcite{Dolan:2002du}. In this context, we implemented a free software that makes Performance Profile using data provided by user in a friendly manner. This software produces graphics in PDF using \LaTeX with PGF/TikZ and \texttt{pgfplots} packages, in PNG using \texttt{matplotlib}
and can also be easily extended to use with other plot library. The software is
implemented in Python3 with support for internationalization. %and is available on \url{https://github.com/lpoo/perprof-py}.
\end{abstract}

\printbibliography
\end{document}