Implementation is separated in a way that facilitates the creation of a
new backend.
Class {\tt Pdata} is defined to store the parsed data ($\mathcal{P}$,
$\mathcal{S}$, $t_{s,p}$, etc.) and  methods are defined to create the profile data
$r_{p,s}$.
Backends are classes that extend {\tt Pdata} defining a method {\tt plot}
which creates the expected figure.
One shall have little difficulty creating his own backend, specially if he
uses one of perprof-py own as a starting point.
However, if a user wants to change the profile data definition --- as
to implement  data profile (see \cite{bib:more2009benchmarking}) ---, he would have to modify
one or more methods in {\tt Pdata} directly or re-implement the backends.

Parser opens the input files and creates the information for {\tt Pdata}.
Replacing this parser --- to use with perprof-py backends --- would not be an easy task
since the correct output format should be created.  Nevertheless,
extending it with additional options would be simple enough.

Entry point {\tt perprof-py} essentially collects the options from the command
line and calls the specific backend profiler. This can be completely bypassed
by calling the backend directly. This allows one to create a performance
profile from another python application. In particular, one possibility is the
creation of a graphical user interface (GUI)
or a web server application. Perprof-py modularity 
allows whoever desires to construct this interface to focus entirely on
obtaining the options from the user and passing it to the backend.

Whether one is planning on expanding some of perprof-py functionalities or creating any
new backend or interface, he can contact the authors using the project page on
GitHub~\cite{url:perprof-py}.

