\subsection*{Operating system}

    Perprof-py is developed and actively tested on Unix platforms, however, it
    is also possible to run it on Windows, removing some features.

\subsection*{Programming language}

    The project was made entirely with python3.

\subsection*{Additional system requirements}

    No additional hardware requirement are necessary.

\subsection*{Dependencies}

    Perprof-py depends on the Python packages matplotlib-1.3.1, nose-1.3.0,
    numpy-1.8.0, pyparsing-2.0.1, python-dateutil-2.2, six-1.4.1, tornado-3.1.1
    and pyyaml. Additionally, if the user wants the PDF image from the LaTeX
    version, it also requires pdflatex.

\Archive

    \subsubsection*{Name:} perprof-py

    \subsubsection*{Identifier:} \url{http://dx.doi.org/10.5281/zenodo.11350}

    \subsubsection*{Licence:} GPL (General Public License) Version 3

    \subsubsection*{Date published:} 20/08/14

    \subsubsection*{Publisher:} Abel Soares Siqueira

    \subsubsection*{Date published:} 24/03/14

\CodeRepository

    \subsubsection*{Name:} GitHub

    \subsubsection*{Identifier:} \url{https://github.com/lpoo/perprof-py}

    \subsubsection*{Licence:} GPL (General Public License) Version 3

    \subsubsection*{Date published:} 24/03/14

\subsubsection*{Language}

    Language of repository, software and supporting files

