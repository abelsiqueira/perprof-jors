\subsection*{Operating system}

    Perprof-py is developed and actively tested on Unix platforms.
    The authors did not test it on Windows.

\subsection*{Programming language}

    The project was built entirely on Python 3.

\subsection*{Additional system requirements}

    No additional hardware requirement are necessary.

\subsection*{Dependencies}

    Perprof-py depends on the Python packages \texttt{matplotlib}, \texttt{pyyaml} and \texttt{bokeh}.
    In addition, if a user wants the PDF image from the \LaTeX\
    version, it also requires \texttt{pdflatex}.

\Archive

    \subsubsection*{Name:} perprof-py v1.1.0

    \subsubsection*{Identifier:} \url{\pubdoi}

    \subsubsection*{Licence:} GPL (General Public License) Version 3

    \subsubsection*{Date published:} 30/05/15

    \subsubsection*{Publisher:} Abel Soares Siqueira

    \subsubsection*{Date published:} 30/05/15

\CodeRepository

    \subsubsection*{Name:} GitHub

    \subsubsection*{Identifier:} \url{https://github.com/ufpr-opt/perprof-py}

    \subsubsection*{Licence:} GPL (General Public License) Version 3

    \subsubsection*{Date published:} 30/05/15

\subsubsection*{Language}

    The entire project was developed in English, but there is support for
    other languages on the code. Currently, the only other language implemented, besides English, is Brazilian Portuguese.

